\section{Session on Low Energy Physics}\label{lowenergy}

\fix{Editors: Olivier, Adrian, Klaus} \fix{Preliminary version}

The Low Energy Physics session consisted of the following three  presentations:
\begin{itemize} \setlength{\itemsep}{-1ex}
\item ``Implications of the flavor anomalies'', by  Andreas Crivellin (PSI);
\item ``Searches on cLFV with muons and time reversal symmetry violation with
neutrons'', by Angela Papa (PSI);
\item ``Precision measurements with ultracold
neutrons, muons, positrons, and antiprotons'', by Andreas Knecht (PSI).
\end{itemize}

The session followed the one on precision heavy Flavour Physics and
started with a bridging presentation by Andreas Crivellin who
reflected on the recent and longer standing deviations of a few sigma
each from the Standard Model in heavy flavour physics and discussed a
few particularly attractive model scenarios to accommodate them. These
could naturally be confronted with highly sensitive observables of low
energy precision physics. To name a few, the muon anomalous magnetic
moment, the charged lepton flavour violating muon decays, the proton
charge radius, and the ratio of decays of charged pions to electrons
versus muons. The following two experimental talks then focused on the
Swiss landscape in low energy precision physics.

Angela Papa's presentation concentrated on aspects of the rare muon
decay searches $\mu\to e \gamma$ and $\mu\to eee$ as well as on the
search for the CP-violating electric dipole moment of the neutron
(nEDM).  The charged lepton flavour violating muon decays are the
overall most sensitive rare decay searches. 
In all three so-called golden channels (the two
mentioned above and the $\mu$-$e$ conversion channel), 
PSI has obtained the current best limits because of its superior muon beams. 
The most recent one was released by the international MEG
collaboration in 2016. While strong collaborations pursue projects to
improve the sensitivity for $\mu$-$e$ conversion at J-PARC and at
FNAL, projects at PSI aim at pushing the searches for $\mu\to e
\gamma$ (MEG-II) and $\mu\to eee$ (Mu3e) by one and 3-4 orders of
magnitude, respectively, in the coming years. Swiss particle physics is
in a pole position to assume leading roles in these experiments (so
far: MEG: PSI, Mu3e: UGe, UZH, ETHZ). Mu3e is planned in phases
and the second phase will need a new High Intensity Muon Beam (HiMB)
to be built at PSI's HIPA facility with the strong support of the Swiss
particle physics community.

The nEDM experiment is currently taking data at PSI's world leading
source of ultracold neutrons (UCN). It will supersede the previous
best result with its 2015/16 data set. The international nEDM
collaboration (CH so far: PSI, ETHZ) will continue its activities with
the new n2EDM experiment which will be set up around 2018 and gain
another order of magnitude in sensitivity.

Andreas Knecht's presentation highlighted some Swiss activities in the
low energy precision field besides the search experiments. The
particles very conveniently available in Switzerland are the positrons
at ETHZ, the antiprotons at CERN and the UCN and muons at PSI. 
The importance of hosting in our country some of the world's best particle
sources icannot be overestimated. The range of fundamental physics questions
which are being addressed with these probes is very broad: a hot topic
is a measurement of the gravitational interaction of antimatter
pursued by various projects with the neutral atoms of anti-hydrogen,
positronium and muonium. All these are, together with a number of
muonic atoms, also targets for high precision spectroscopy aiming at
tests of QED and of the CPT symmetry, and at the determination of fundamental
constants, masses as well as charge and magnetic radii of nucleons and light
nuclei. UCN and the leptons are also being used in sensitive searches
for exotic interactions, axions, axion-like particles, mirror matter
and other dark matter candidates. These are complementarily covering
regions of parameter space often not accessible to collider
experiments or direct dark matter detection. Of particular importance
are future improvements of the particle sources and Andreas Knecht
emphasized the development of a very effective low energy muon cooling
at PSI as well as the HiMB mentioned above already.

From the related discussions, one can conclude that the searches for
cLFV with muons and for CPV with neutrons offer unique opportunities
and considerable discovery potential. The low energy precision physics
program and, in particular, an upgrade of the world-leading high
intensity infrastructure at PSI require an active participation of the
CHIPP community.
