\section{Day 2 Summary: Flavor and Low Energy Summary and Connection to other 
fields}\label{discussionconnetion}

\fix{editors: Gino, Olivier}



The discussion covered three main subjects: I) the prospects for indirect NP searches, mainly via flavour-physics 
measurements, within the LHC experiments (LHCb, but also ATLAS and CMS);
II) the connections between low- (non-LHC) and high-energy physics experiments within the first pillar, with particular attention to the role of PSI;
III) the connections between the first pillar and the other two pillars.
The main points of discussions on these three main subjects can be summarised as follows

\begin{enumerate}
\item[I.a] {\em Flavour physics and indirect NP searches at ATLAS and CMS.} 
The question has been raised if ATLAS \& CMS should consider possible 
modifications of their upgrade plans (for the HL phase) in view of present flavour anomalies,
especially in the absence of direct signals of NP. These anomalies, and also various theoretical 
arguments, seem to suggest a particular interest in tau- and b-quark enriched final states
and, more generally, in an optimal flavour-tagging (and flavour-discrimination) efficiency. 
After an extensive discussion, a consensus was reach on the fact that this request
is already well addressed by the present upgrade plans of ATLAS \& CMS,
taking into account also the fact that these high-PT experiments 
must optimize the NP sensitivity in all possible directions.

\item[I.b] {\em  LHCb upgrade}. 
Motivated by the strong interest in an extension of the $B$-physics program at the LHC,
advocated in various theory talks at the meeting and reinforced by the recent interesting results in this sector, 
the possibility of a further upgrade of the LHCb experiment  has been discussed.
The already planned LHCb upgrade aims to collect 50$fb^{-1}$ by 20xx.
In principle, a luminosity $\sim 40$ times higher would be available in the HL phase of the 
LHC. Can a further-upgraded LHCb stand such a luminosity (or a significant fraction of it)? 
Beside the maximal luminosity,  can some of the present LHCb performances (e.g. on electron and tau modes) 
be increased in view of a further upgrade? 
The LHCb Collaboration is considering this interesting option, but a detailed answer on its feasibility requires time.


\item[II] {\em Low-energy physics}. As outlined in the theory talks, the indirect NP searches performed at low-energies 
are extremely interesting and, to a large extent, independent from the direct searches performed at high-energies. 
This point was further emphasised during the discussion session, with particular attention to the PSI program. 
In particular, it was stressed the importance to secure (both in terms of budget and in terms of man power) 
the interesting and ambitious  $\mu \to 3 e$ phase-II program, independently of the developments 
at the high-energy frontier. The point was raised that a potential firm evidence of lepton-flavour non-universality
in $B$ decays would, on general grounds, render the physics case of  CLFV searches 
in $\mu$ decays even stronger.  However, it was also concluded that the opposite is not true 
(CLFV in  $\mu$  decays could occur independently of LFU in B decays), since the connections between these 
two sectors is very model dependent.


\item[III.a] {\em Connections with the neutrino program}\\
  Q:	Does the evolution of the neutrino program influence the future HEP
  	program? \\
  A: The physics goals are connected, but the two programs run essentially 
	in parallel.

\item[III.b] {\em Connections with the DM program (direct \& indirect searches)}\\
  Q:	Does the evolution of the neutrino program influence the future 
	HEP program? \\
 A:	There are potential connections but these are model dependent 
	($\to$ worth planning joint analyses in case of positive signals, 
	either on the DM or on the HEP side).
	No clear influence of DM searches on the HEP program at present, 		
  	the situation may change with clear evidences of light DM candidates.  

